\documentclass[12pt,a4paper]{article}
\usepackage[UTF8]{ctex}  % 中文支持
\usepackage{xlop}        % 竖式计算
\usepackage{geometry}    % 页面设置
\usepackage{multicol}    % 多列布局
\usepackage{xcolor}      % 颜色
\usepackage{fancyhdr}    % 页眉页脚

% 页面设置
\geometry{
  a4paper,
  left=2cm,
  right=2cm,
  top=2.5cm,
  bottom=2cm
}

% 页眉页脚
\pagestyle{fancy}
\fancyhf{}
\renewcommand{\headrulewidth}{0.4pt}
\fancyhead[C]{\color{blue}\Large\textbf{\VAR{title}}}
\fancyfoot[C]{\thepage}

% 自定义命令
\newcommand{\sectiontitle}[1]{
  \vspace{0.5cm}
  \noindent\colorbox{blue!10}{\parbox{\textwidth}{\color{blue!80!black}\large\textbf{#1}}}
  \vspace{0.3cm}
}

\begin{document}

% 标题和信息栏
\begin{center}
  {\color{blue}\Huge\textbf{\VAR{title}}}
\end{center}

\vspace{0.3cm}

\noindent
姓名:\underline{\hspace{3cm}} \hfill
班级:\underline{\hspace{3cm}} \hfill
日期:\underline{\hspace{3cm}} \hfill
成绩:\underline{\hspace{3cm}}

\vspace{0.5cm}

\BLOCK{if oral_questions}
% 一、口算题
\sectiontitle{一、口算题(每题\VAR{oral_points}分,共\VAR{oral_questions|length * oral_points}分)}

\begin{multicols}{2}
\BLOCK{for q in oral_questions}
  \VAR{loop.index}. \VAR{q.question}

  \vspace{0.3cm}
\BLOCK{endfor}
\end{multicols}
\BLOCK{endif}

\BLOCK{if vertical_questions}
% 二、竖式计算
\sectiontitle{二、竖式计算(每题\VAR{vertical_points}分,共\VAR{vertical_questions|length * vertical_points}分)}

\begin{multicols}{2}
\BLOCK{for q in vertical_questions}
  % 题号和边框
  \noindent
  \fbox{
    \begin{minipage}[t][5cm][t]{6cm}
      % 题号
      \textbf{\VAR{loop.index}}

      \vspace{1cm}

      % 竖式内容
      \begin{center}
      \BLOCK{if q.operation == 'add' or q.operation == 'sub'}
        % 加减法竖式
        \Large
        \begin{tabular}{@{}r@{}}
          \VAR{q.numbers.0} \\
          \BLOCK{if q.operation == 'add'}+\BLOCK{else}-\BLOCK{endif} \VAR{q.numbers.1} \\
          \hline
          \phantom{0000} \\
        \end{tabular}
      \BLOCK{elif q.operation == 'mul'}
        % 乘法竖式
        \Large
        \begin{tabular}{@{}r@{}}
          \VAR{q.numbers.0} \\
          $\times$ \VAR{q.numbers.1} \\
          \hline
          \phantom{0000} \\
        \end{tabular}
      \BLOCK{elif q.operation == 'div'}
        % 除法竖式 - 教材标准格式
        \Large
        \setlength{\unitlength}{1mm}
        \begin{picture}(50,25)
          % 商的格子区域(横线上方,每位数一个格子)
          \put(15,18){\framebox(8,6){}}
          \put(23,18){\framebox(8,6){}}
          \put(31,18){\framebox(8,6){}}

          % 横线
          \put(15,18){\line(1,0){30}}

          % 竖线
          \put(15,3){\line(0,1){15}}

          % 除数(竖线左边)
          \put(13,10){\makebox(0,0)[r]{\Large\VAR{q.numbers.1}}}

          % 被除数(竖线右边)
          \put(20,10){\makebox(0,0)[c]{\Large\VAR{q.numbers.0}}}

          % 计算区域留白
          \put(17,3){\makebox(0,0)[c]{\phantom{计算}}}
        \end{picture}
      \BLOCK{endif}
      \end{center}
    \end{minipage}
  }

  \vspace{0.5cm}
\BLOCK{endfor}
\end{multicols}
\BLOCK{endif}

\BLOCK{if fill_questions}
% 三、填空题
\sectiontitle{三、填空题(每题\VAR{fill_points}分,共\VAR{fill_questions|length * fill_points}分)}

\BLOCK{for q in fill_questions}
  \VAR{loop.index}. \VAR{q.question}

  \vspace{0.4cm}
\BLOCK{endfor}
\BLOCK{endif}

\BLOCK{if list_vertical_questions}
% 四、列竖式计算
\sectiontitle{四、列竖式计算(每题\VAR{list_points}分,共\VAR{list_vertical_questions|length * list_points}分)}

\BLOCK{for q in list_vertical_questions}
  \VAR{loop.index}. \VAR{q.question}(请在下方列竖式计算)

  \vspace{2cm}
\BLOCK{endfor}
\BLOCK{endif}

% 答案页
\newpage

\begin{center}
  {\color{green!70!black}\Huge\textbf{参考答案}}
\end{center}

\vspace{0.5cm}

\BLOCK{if oral_questions}
\sectiontitle{一、口算题}
\begin{multicols}{5}
\BLOCK{for q in oral_questions}
  \VAR{loop.index}. \VAR{q.answer}

\BLOCK{endfor}
\end{multicols}
\BLOCK{endif}

\BLOCK{if vertical_questions}
\sectiontitle{二、竖式计算}
\begin{multicols}{5}
\BLOCK{for q in vertical_questions}
  \VAR{loop.index}. \VAR{q.answer}

\BLOCK{endfor}
\end{multicols}
\BLOCK{endif}

\BLOCK{if fill_questions}
\sectiontitle{三、填空题}
\begin{multicols}{5}
\BLOCK{for q in fill_questions}
  \VAR{loop.index}. \VAR{q.answer}

\BLOCK{endfor}
\end{multicols}
\BLOCK{endif}

\BLOCK{if list_vertical_questions}
\sectiontitle{四、列竖式计算}
\begin{multicols}{5}
\BLOCK{for q in list_vertical_questions}
  \VAR{loop.index}. \VAR{q.answer}

\BLOCK{endfor}
\end{multicols}
\BLOCK{endif}

\end{document}
